\documentclass{report}
\usepackage{pgfplots}
\usepackage{geometry}
\usepackage{fancyhdr}
\usepackage{braket}
\usepackage{amsmath}
\usepackage{graphicx} % Required for inserting images
\usepackage[margin=1in]{geometry}
% Customize headers
\pagestyle{fancy}
\fancyhf{}
\renewcommand{\headrulewidth}{2pt}
\title{Conceptual Review : Logical Clifford Synthesis for Stabilizer Codes}
\author{Aparna Gupta}
\date{February 2024}

\begin{document}

\maketitle

\section*{Introduction}    

The algorithm in this paper reduces the stabilizer matrix of the code into a standard form to determine the logical Pauli operators. For a stabilizer Quantum error correcting code, the physical results of applying a Clifford operator on the logical qubits can be represented as a $2m \times 2m$ matrix, reducing the computational complexity from $O(2^m)$ to $O(2m)$. This algorithm assembles all symplectic matrices, of Clifford and of Pauli operator, to realize the unitary logical operator being implemented on the system.

\section*{Pauli Group and Clifford Group}

\subsection*{Pauli Group}

Pauli operators acting on an n-qubit system can be defined as the Pauli group such that,
\[ P_{n} = \{I,X,Y,Z\}^{\otimes n} \]
where $I$, $X$, $Y$, and $Z$ are Pauli matrices defined as:
\begin{enumerate}
    \item $I = \begin{bmatrix} 1 & 0 \\ 0 & 1 \end{bmatrix}$
    \item $X = \begin{bmatrix} 0 & 1 \\ 1 & 0 \end{bmatrix}$
    \item $Z = \begin{bmatrix} 1 & 0 \\ 0 & -1 \end{bmatrix}$
    \item $Y = iXZ = \begin{bmatrix} 0 & -i \\ i & 0 \end{bmatrix}$
\end{enumerate}

\subsection*{Clifford Group}

The Clifford group on $N$ qubits is defined as the normalizer of the Pauli group in the $U_{N}$, i.e., 
\[ C_{N} = \{ U \in U_{N} : UPU^\dag = P_{N} \} \]
where $P_{N}$ is the group generated by Pauli operators $I$, $Y$, $Z$, and $X$, acting on each qubit.

\section*{Symplectic Geometry}

\subsection*{Heisenberg-Weyl Pauli Group}

Given row vectors $a = \{a_{1}, \ldots, a_{m}\}$ and $b = \{b_{1}, \ldots, b_{m}\} \in F^{2}_{m}$, we define the $m$-qubit operator as,
\[ D(a,b) = X^{a_{1}}Z^{b_{1}} \otimes \ldots \otimes X^{a_{m}}Z^{b_{m}} \]

Using this definition, we can say that the Heisenberg-Weyl Pauli Group on $m$ qubits is defined as,
\[ HW_{N} = \{i^\kappa D(a,b) \mid a,b \in F^{2}_{m}, \kappa \in \{0,1,2,3,4\}\} \]

Elements of $HW_{N}$ satisfy the following properties:
\begin{enumerate}
    \item $D(a,b)D(a',b') = (-1)^{a'b^{T} + b'a^{T}} D(a',b')D(a,b)$
    \item $D(a,b)^T = (-1)^{ab^{T}}D(a,b)$
    \item $D(a,b)D(a',b') = (-1)^{a'}b^{T}D(a+a',b+b')$
\end{enumerate}

The Clifford group is the normalizer of $HW_{N}$ in the unitary group $U_{N}$, i.e., Cliff_{N} = \mathbf{N_{U_{N}}}{(HW_{N})}.
 \subsubsection*{Examples}
Let us consider a two qubit system,
\begin{enumerate}
    \item implement X gate on the first qubit:
    \[ D(a,b) = X^{1}Z^{0} \otimes X^{0}Z^{0} \rightarrow \begin{bmatrix}
    1 & 0 \vdots 0 & 0
    \end{bmatrix}
    \]
    \item implement X gate on the second qubit:
    \[ D(a,b) = X^{0}Z^{0} \otimes X^{1}Z^{0} \rightarrow \begin{bmatrix}
    0 & 1 \vdots 0 & 0
    \end{bmatrix}\]
    \item implement Z gate on the first qubit:
    \[ D(a,b) = X^{0}Z^{1} \otimes X^{0}Z^{0} \rightarrow \begin{bmatrix}
    0 & 0 \vdots 1 & 0
    \end{bmatrix}\]
    \item implement Z gate on the second qubit:
    \[ D(a,b) = X^{0}Z^{0} \otimes X^{0}Z^{1} \rightarrow \begin{bmatrix}
    0 & 0 \vdots 0 & 1
    \end{bmatrix}\]
\end{enumerate}

\subsection*{Symplectic Inner Product}

The symplectic inner product is defined as,
\[ \braket{[a,b],[a',b']} = a'b^{T} + b'a^{T} = [a,b]\Omega [a',b']^T \]
where the symplectic form in $F^{2m}_{m}$ is $\Omega = \begin{bmatrix} 0 & I_{m} \\ I_{m} & 0 \end{bmatrix}$.

$D(a,b)$ and $D(a^T,b')$ commute if and only if $\braket{[a,b],[a',b']} = 0$.

\subsection*{Symplectic Matrices}
\subsubsection*{Examples}

We will construct the symplectic matrix for a CX (Controlled Not) gate, CZ (Controlled Phase) gate, and Hadamard gate.
\begin{enumerate}
    \item Let us try to derive the symplectic matrix for the Hadamard gate. We know,
\[HXH = Z\ \Rightarrow  \begin{bmatrix}
    1 & 0
\end{bmatrix}\rightarrow \begin{bmatrix}
    0 & 1
\end{bmatrix}\]
Simillarly,
\[HZH = X \Rightarrow \begin{bmatrix}
    0 & 1
\end{bmatrix}\rightarrow \begin{bmatrix}
    1 & 0
\end{bmatrix}\]
Therefore, the symplectic matrix for the Hadamard gate can be written as,
\[F_{H} = \begin{bmatrix}
    0 & 1\\1 & 0
\end{bmatrix}\]
\item If we implement a controlled-not gate on a two qubits system, then we can expect to get a final state,
\[ \ket{\psi} = \ket{0}\bra{0} \otimes I + \ket{1}\bra{1} \otimes X\]

Now let us consider a two qubit system,
\[CX(X_{1}\otimes I_{2})CX = X_{1}X_{2} \Rightarrow  \begin{bmatrix}
    1 & 0 \vdots 0 & 0
    \end{bmatrix} \rightarrow \begin{bmatrix}
    1 & 1 \vdots 0 & 0
    \end{bmatrix}\]
\[CX((I_{1}\otimes X_{2} )CX = I_{1}X_{2} \Rightarrow  \begin{bmatrix}
    0 & 1 \vdots 0 & 0
    \end{bmatrix} \rightarrow \begin{bmatrix}
    0 & 1 \vdots 0 & 0
    \end{bmatrix}\]
\[CX(Z_{1}\otimes I_{2})CX = Z_{1}I_{2} \Rightarrow  \begin{bmatrix}
    0 & 0 \vdots 1 & 0
    \end{bmatrix} \rightarrow \begin{bmatrix}
    0 & 0 \vdots 1 & 0
    \end{bmatrix}\]
\[CX((I_{1}\otimes Z_{2} )CX = I_{1}Z_{2} \Rightarrow  \begin{bmatrix}
    0 & 0 \vdots 0 & 1
    \end{bmatrix} \rightarrow \begin{bmatrix}
    0 & 0 \vdots 0 & 1
    \end{bmatrix}\]
Therefore, the symplectic matrix for the CX gate can be written as,
\[F_{CX} = \begin{bmatrix}
    1 & 1 & 0 & 0\\0 & 1 & 0 & 0 \\0 & 0 & 1 & 0\\0 & 0 & 0 & 1
\end{bmatrix}\]
\end{enumerate}
\subsection*{Symplectic Groups}\
\subsection*{Symplectic Basis}\
\subsection*{Automorphism}\
\subsection*{Symplectic Transvections}\
\section*{Generic Algorithm for Synthesis of Logical Clifford Operators}


\end{document}
